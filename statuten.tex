\documentclass[a4paper,12pt]{article}
\input{template.lco}

% Titel
\title{Statuten des Vereins}
\author{Cyber Perikarp - Verein z. Förderung d. Netzkultur}
\date{\today}

% Funktion für Zitate
\newcommand*{\zitat}[2]{%
   \normalfont
	 \small
   \begin{quote} ,,#1``\par
   	\textit{#2}
   \end{quote}
   \normalsize
}

% Funktionsperiode (damit man sie einfach ändern kann)
\newcommand{\funktionsperiode}{zwei Jahre }

\begin{document}
	% Titel
	\maketitle
	% Text start
	\section{Präambel}
		\label{sec:vorwort}
		\zitat{Jeder hat das Recht auf Meinungsfreiheit und freie Meinungsäusserung; dieses Recht schliesst die Freiheit ein, Meinungen ungehindert anzuhängen sowie über Medien jeder Art und ohne Rücksicht auf Grenzen Informationen und Gedankengut zu suchen, zu empfangen und zu verbreiten.}{Artikel 19 der Allgemeinen Erklärung der Menschenrechte, 1948}
		Der Zugang zu und Austausch durch Medien ist integraler Bestandteil einer demokratischen Werteordnung und eine Gesellschaft ohne Computer nicht mehr vorstellbar. Datenübermittlung bringt viele Vorteile, aber auch Gefahren mit sich. Der Verein setzt sich das Ziel, die Menschen über die Möglichkeiten digitaler Netzwerke aufzuklären und ihnen untersetützend zur Seite zur stehen. Der Verein setzt sich als Ziel aktiv gegen Überwachung und Spionage einzutreten. Dies äussert sich unter anderem im Betrieb mehrerer Server für das Tor Netzwerk, eines Freemailingdienstes und in der aktiven Aufklärung der Bevölkerung über (staatliche) Überwachung und Massnahmen zur Förderung der Privatsphäre.

	\section{Name und Sitz}
		\label{sec:name}
		\begin{enumerate}
			\item Der Verein führt den Namen ,,Cyber Perikarp - Verein zur Förderung der Netzkultur``, abgekürzt ,,Cyber Perikarp``.
			\item Der Verein hat seinen Sitz in Krieglach und erstreckt seine Tätigkeit auf das gesamte Bundesgebiet. Unter Berücksichtigung der technischen Möglichkeiten elektronischer Netzwerke erstreckt der Verein seine Tätigkeit auf die gesamte Welt.
			\item Er ist unpolitisch und überparteilich sowie gemeinnützig.
			\item Die Errichtung von Zweigvereinen ist nicht beabsichtigt.
			\item Das Geschäftsjahr ist das Kalenderjahr.
		\end{enumerate}

	\section{Vereinszweck}
		\label{sec:zweck}
		Die Tätigkeit des Vereins ist nicht auf Gewinn ausgerichtet, er verfolgt ausschliesslich gemeinnützige
		Ziele. Die aus seiner Tätigkeit entstehenden Erträgnisse und Überschüsse dürfen nur für
		satzungsgemässe Zwecke verwendet werden.\par
		Ziele des Vereins sind insbesondere:
		\begin{enumerate}
			\item Öffentlichkeitsarbeit in Form der Herausgabe von regelmässigen Informationen in analoger und digitaler Form inklusive
			Internetauftritt,
			\item Kunst und Kultur in elektronischen Netzwerken zu fördern,
			\item Ein elektronisches Netzwerk zu errichten und zu betreiben,
			\item Forschung, Entwicklung und Umsetzung neuer Technologien,
			\item Anbieten von Diensten und Dienstleistungen in elektronischen Netzwerken und Förderung dieser,
			\item Veranstaltungen durchzuführen,
			\item Informationsaustausch mit den in der Datenschutzgesetzgebung vorgesehenen Kontrollorganen,
			\item Arbeits- und Erfahrungsaustauschkreise,
			\item Hilfestellung und Beratung bei technischen und rechtlichen Fragen im Rahmen der gesetzlichen Möglichkeiten,
			\item Bewerbung von und Informationen über elektronische Netzwerke,
			\item Errichtung und Erhaltung von Klubräumen, Laboratorien und Büchereien sowie Herausgabe von Informationsmaterial,
			\item Förderung der Privatsphäre von Anwendern in elektronischen Netzwerken und Information der Anwender über Möglichkeiten, diese zu erreichen,
			\item Herausgabe von Pressemitteilungen und Öffentlichkeitsarbeit,
			\item Die Herstellung von Verbindungen und Informationsaustausch mit gleichgesinnten Vereinen im In- und Ausland.
			\item Der Verein fördert und unterstützt Vorhaben der Bildung und Volksbildung, die sich dem Vereinszweck widmen, insbesondere der anonymen, sicheren und verschlüsselten elektronischen Kommunikation.
		\end{enumerate}
		Hinzu kommt als unterstützendes ideelles Mittel die Bereitstellung der technischen Infrastruktur (Mailinglisten, Webserver, etc.) zur Kommunikation zwischen den Vereinsmitgliedern untereinander, mit anderen (internationalen) Vereinigungen mit ähnlichen Zielsetzungen und der allgemeinen Öffentlichkeit.

	\section{Mittel zur Erreichung des Vereinszweckes}
		\label{sec:mittel}
		Die erforderlichen finanziellen Mittel sollen aufgebracht werden durch
		\begin{enumerate}
			\item Beitrittsgebühren,
			\item Mitgliedsbeiträge,
			\item Subventionen und Förderungen,
			\item Erlöse aus Veranstaltungen,
			\item Vermögensverwaltung (z.B. Zinsen, sonstige Kapitaleinkünfte, Einnahmen aus Vermietung und Verpachtung usw.),
			\item Spenden, Sammlungen, Vermächtnisse und sonstige Zuwendungen,
			\item Sponsoring,
			\item Werbeeinnahmen,
			\item sonstige Zuwendungen und andere legale Einnahmen im Rahmen der Möglichkeiten.
		\end{enumerate}

	\section{Arten der Mitgliedschaft}
		\label{sec:arten-mitgliedschaft}
		Die Mitglieder des Vereins gliedern sich in:
		\begin{enumerate}
			\item Ordentliche Mitglieder, das sind jene, die aktiv an der Erreichung des Vereinszweckes mitarbeiten und das Vereinsleben mitgestalten,
			\item Gründungsmitglieder, das sind all jene, die an der Gründungsversammlung am 24.09.2016 teilgenommen haben,
			\item Ehrenmitglieder, das sind jene, die auf Grund besonderer Verdienste für den Verein von der Generalversammlung dazu ernannt wurden,
			\item Unterstützende Mitglieder, das sind jede, die durch ihren Mitgliedsbeitrag den idellen Zweck des Vereins unterstützen.
		\end{enumerate}

	\section{Aufnahme von Mitgliedern}
		\label{sec:aufname-mitglieder}
		\begin{enumerate}
			\item Mitglieder des Vereins können alle physischen und juristischen Personen, Personengesellschaften sowie Personengruppen werden, die die Statuten anerkennen und den Vereinszweck fördern wollen.
			\item Personengruppen sind zum Beispiel Interessensgemeinschaften, KünstlerInnenkollektive und ähnliche Gruppen ohne eigene Rechtspersönlichkeit. Für Personengruppen und Personengesellschaften gelten analog die Bestimmungen wie für juristische Personen.
			\item Die ordentliche oder unterstützende Mitgliedschaft ist mit einem schriftlichen Beitrittsansuchen an den Vorstand zu beantragen.
			\item Bis zur Entstehung des Vereins erfolgt die vorläufige Aufnahme von ordentlichen und unterstützenden Mitgliedern durch die die Vereinsgründer, im Fall eines bereits bestellten Vorstands durch diesen. Diese Mitgliedschaft wird erst mit Entstehung des Vereins wirksam. Wird ein Vorstand erst nach Entstehung des Vereins bestellt, erfolgt auch die (definitive) Aufnahme ordentlicher und ausserordentlicher Mitglieder bis dahin durch die Gründer des Vereins.
			\item Juristische Personen haben schriftlich einen Vertreter zu bestimmen, der deren Interessen im Verein wahrnimmt. Jede juristische Person kann nur einen Vertreter bestimmen. Die Bestimmung eines Vertreters gilt ein Jahr oder bis auf Widerruf. Das Präsidium kann ohne Begründung die Bestimmung eines Vertreters ablehnen und die juristische Person auffordern, einen anderen Vertreter zu bestimmen. Solche Vertreter geniessen das aktive und passive Wahlrecht an Stelle der von ihnen vertretenen juristischen Person, sofern diese ein ordentliches Mitglied oder Gründungsmitglied ist.
			\item Die Ernennung zum Ehrenmitglied erfolgt auf Antrag des Vorstands durch die Generalversammlung.
			\item Der Verein erhebt eine Beitragsgebühr, deren Höhe von der Generalversammlung festgelegt wird.
			\item Der Antrag auf Mitgliedschaft kann ohne Begründung abgelehnt werden.
		\end{enumerate}

	\section{Beendigung der Mitgliedschaft}
		\label{sec:ende-mitgliedschaft}
		\begin{enumerate}
			\item Die Mitgliedschaft erlischt durch Tod bei juristischen Personen und rechtsfähigen Personengesellschaften durch Verlust der Rechtspersönlichkeit, durch freiwilligen Austritt, durch Streichung oder durch Ausschluss.
			\item Der Austritt kann nur zum ersten Tag eines jedes Monats erfolgen. Er muss dem Vorstand mindestens 14 Tage vorher mitgeteilt werden. Erfolgt die Anzeige verspätet, so ist sie erst zum nächsten Austrittstermin
			wirksam.
			\item Der Vorstand kann ein Mitglied ausschliessen, wenn dieses trotz zweimaliger schriftlicher Mahnung unter Setzung einer angemessenen Nachfrist länger als sechs Monate mit der Zahlung der Mitgliedsbeiträge im Rückstand ist. Die Verpflichtung zur Zahlung der fällig gewordenen Mitgliedsbeiträge bleibt hiervon unberührt.
			\item Der Ausschluss eines Mitglieds aus dem Verein kann vom Vorstand auch wegen grober Verletzung anderer Mitgliedspflichten und wegen unehrenhaften Verhaltens verfügt werden.
			\label{abs:ausschluss}
			\item Die Aberkennung der Ehrenmitgliedschaft kann aus den im Abs. \ref{abs:ausschluss} genannten Gründen von der Generalversammlung über Antrag des Vorstands beschlossen werden.
		\end{enumerate}

	\section{Rechte und Pflichten der Mitglieder}
		\label{sec:rechte-pflichten}
		\begin{enumerate}
			\item Die Mitglieder sind nach den vereinsüblichen Regelungen berechtigt, an allen Veranstaltungen des Vereins teilzunehmen und die Einrichtungen des Vereins zu beanspruchen.
			\item Das Stimmrecht in der Generalversammlung sowie das aktive und passive Wahlrecht stehen allen ordentlichen Mitgliedern sowie den Gründungsmitgliedern zu.
			\item Die Mitglieder sind verpflichtet, die Interessen des Vereins nach Kräften zu fördern und alles zu unterlassen, worunter das Ansehen und der Zweck des Vereins leiden könnten. Sie haben die Vereinsstatuten, die Geschäftsordnung und die Beschlüsse der Vereinsorgane zu beachten. Sie sind zur pünktlichen Zahlung der Mitgliedsbeiträge in der von der Generalversammlung beschlossenen Höhe verpflichtet.
			\item Jedes Mitglied ist berechtigt, vom Vorstand die Ausfolgung der Statuten zu verlangen.
			\item Mindestens ein Zehntel der Mitglieder kann vom Vorstand die Einberufung einer Generalversammlung verlangen.
			\item Alle Mitglieder haben das Recht, der Generalversammlung und dem Vorstand Anträge zu unterbreiten.
			\item Unterstützende Mitglieder haben einen frei wählbaren Beitrag, mindestens jedoch \EUR{10} pro Jahr, zu entrichten.
			\item Unterstützende Mitglieder können auf Antrag ebendso ordentliche Mitglieder sein.
			\item Gründungsmitglieder geniessen ein Vetorecht gegen alle Beschlüsse der Generalversammlung, die die grundsätzliche Ausrichtung des Vereins betreffen. Dies gilt insbesondere bei Beschlüssen zur Statutenänderung. Dieses Vetorecht kann dergestalt ausgeübt werden, dass mindestens die Hälfte der anwesenden Gründungsmitglieder ein solches Veto gutheissen. Die Gründungsmitglieder haben zu diesem Zwecke das Recht, eine Versammlung zu unterbrechen und sich zur Beratung zurückzuziehen. Eine solche Unterbrechung kann von einem Gründungsmitglied gefordert werden.
		\end{enumerate}

	\section{Organe des Vereins}
		\label{sec:organe}
		Organe des Vereins sind die Generalversammlung (§§ \ref{sec:generalversammlung}, \ref{sec:aufgaben-generalversammlung}), der Vorstand (§§ \ref{sec:vorstand}, \ref{sec:aufgaben-vorstand}), die Rechnungsprüfer (§ \ref{sec:rechnungspruefer}) und das Schiedsgericht (§ \ref{sec:schiedsgericht}).

	\section{Generalversammlung}
		\label{sec:generalversammlung}
		\begin{enumerate}
			\item Die Generalversammlung ist die ,,Mitgliederversammlung`` im Sinne des Vereinsgesetzes 2002. Eine ordentliche Generalversammlung findet alle \funktionsperiode statt.
			\label{abs:ordentliche-generalversammlung}
			\item Eine ausserordentliche Generalversammlung findet auf
			\label{abs:ausserordengliche-generalversammlung}
			\begin{enumerate}
				\item Beschluss des Vorstands oder der ordentlichen Generalversammlung,
				\label{abs:ausserordentlich-vorstand}
				\item schriftlichen Antrag von mindestens einem Zehntel der Mitglieder, Verlangen der Rechnungsprüfer (§ 21 Abs. 5 erster Satz VereinsG),
				\label{abs:ausserordentlich-mitglieder}
				\item Beschluss der/eines Rechnungsprüfer/s (§ 21 Abs. 5 zweiter Satz VereinsG, § \ref{sec:rechnungspruefer} Abs. \ref{abs:rechnungspruefer-kontrolle} dieser Statuten),
				\label{abs:ausserordentlich-rechnung}
				\item Beschluss eines gerichtlich bestellten Kurators (§ \ref{sec:rechnungspruefer} Abs. \ref{abs:rechnungspruefer-kontrolle} letzter Satz dieser Statuten) binnen vier Wochen statt.
				\label{abs:ausserordentlich-kurator}
			\end{enumerate}
			\item Sowohl zu den ordentlichen wie auch zu den ausserordentlichen Generalversammlungen sind alle Mitglieder mindestens zwei Wochen vor dem Termin per Email (an die vom Mitglied dem Verein bekanntgegebene Emailadresse) einzuladen. Die Anberaumung der Generalversammlung hat unter Angabe der Tagesordnung zu erfolgen. Die Einberufung erfolgt durch den Vorstand (Abs. \ref{abs:ordentliche-generalversammlung} und Abs. \ref{abs:ausserordengliche-generalversammlung}  lit. \ref{abs:ausserordentlich-vorstand} - \ref{abs:ausserordentlich-kurator}), durch die/einen Rechnungsprüfer (Abs. \ref{abs:ausserordengliche-generalversammlung} lit. \ref{abs:ausserordentlich-rechnung}) oder durch einen gerichtlich bestellten Kurator (Abs. \ref{abs:ausserordengliche-generalversammlung} lit. \ref{abs:ausserordentlich-kurator}).
			\item Anträge zur Generalversammlung sind mindestens drei Tage vor dem Termin der Generalversammlung beim Vorstand per Email einzureichen.
			\item Gültige Beschlüsse --- ausgenommen solche über einen Antrag auf Einberufung einer ausserordentlichen Generalversammlung --- können nur zur Tagesordnung gefasst werden.
			\item Bei der Generalversammlung sind alle Mitglieder teilnahmeberechtigt. Stimmberechtigt sind nur die ordentlichen und die Ehrenmitglieder. Jedes Mitglied hat eine Stimme. Juristische Personen werden durch einen Bevollmächtigten vertreten. Die Übertragung des Stimmrechts auf ein anderes Mitglied im Wege einer schriftlichen Bevollmächtigung ist zulässig.
			\item Die Generalversammlung ist ohne Rücksicht auf die Anzahl der Erschienenen beschlussfähig.
			\item Die Wahlen und die Beschlussfassungen in der Generalversammlung erfolgen in der Regel mit einfacher Mehrheit der abgegebenen gültigen Stimmen. Beschlüsse, mit denen das Statut des Vereins geändert oder der Verein aufgelöst werden soll, bedürfen jedoch einer qualifizierten Mehrheit von zwei Dritteln der abgegebenen gültigen Stimmen.
			\item Den Vorsitz in der Generalversammlung führt der Obmann in dessen Verhinderung ihre sein Stellvertreter. Wenn auch dieser verhindert ist, so führt das an Jahren älteste anwesende Vorstandsmitglied den Vorsitz.
		\end{enumerate}

	\section{Aufgaben der Generalversammlung}
		\label{sec:aufgaben-generalversammlung}
		Der Generalversammlung sind folgende Aufgaben vorbehalten:
		\begin{enumerate}
			\item Beschlussfassung über den Voranschlag,
			\item Entgegennahme und Genehmigung des Rechenschaftsberichts und des Rechnungsabschlusses unter Einbindung der Rechnungsprüfer,
			\item Wahl und Enthebung der Mitglieder des Vorstands und der Rechnungsprüfer,
			\item Genehmigung von Rechtsgeschäften zwischen Rechnungsprüfer und Verein,
			\item Entlastung des Vorstands,
			\item Festsetzung der Höhe der Beitrittsgebühr und der Mitgliedsbeiträge für ordentliche und für ausserordentliche Mitglieder,
			\item Verleihung und Aberkennung der Ehrenmitgliedschaft,
			\item Beschlussfassung über Statutenänderungen und die freiwillige Auflösung des Vereins,
			\item Beratung und Beschlussfassung über sonstige auf der Tagesordnung stehende Fragen.
		\end{enumerate}

	\section{Vorstand}
		\label{sec:vorstand}
		\begin{enumerate}
			\item Der Vorstand besteht aus drei bis sechs Mitgliedern, und zwar aus Obmann (und Stellvertreter), Schriftführer (und Stellvertreter) sowie Kassier (und Stellvertreter).
			\item Der Vorstand wird von der Generalversammlung gewählt. Der Vorstand hat bei Ausscheiden eines gewählten Mitglieds das Recht, an seine Stelle ein anderes wählbares Mitglied zu kooptieren, wozu die nachträgliche Genehmigung in der nächstfolgenden Generalversammlung einzuholen ist. Fällt der Vorstand ohne Selbstergänzung durch Kooptierung überhaupt oder auf unvorhersehbar lange Zeit aus, so ist jeder Rechnungsprüfer verpflichtet, unverzüglich eine ausserordentliche Generalversammlung zum Zweck der Neuwahl eines Vorstands einzuberufen. Sollten auch die Rechnungsprüfer handlungsunfähig sein, hat jedes ordentliche Mitglied, das die Notsituation erkennt, unverzüglich die Bestellung eines Kurators beim zuständigen Gericht zu beantragen, der umgehend eine ausserordentliche Generalversammlung einzuberufen hat.
			\label{abs:vorstand-wahl}
			\item Die Funktionsperiode des Vorstands beträgt \funktionsperiode und die Wiederwahl ist möglich. Jede Funktion im Vorstand ist persönlich auszuüben.
			\label{abs:vorstand-funktionsperiode}
			\item Der Vorstand wird von vom Obmann, bei Verhinderung von seinem Stellvertreter, schriftlich oder mündlich einberufen. Ist auch dieser auf unvorhersehbar lange Zeit verhindert, darf jedes sonstige Vorstandsmitglied den Vorstand einberufen.
			\item  Der Vorstand ist beschlussfähig, wenn alle seine Mitglieder eingeladen wurden und mindestens die Hälfte von ihnen anwesend ist.
			\item Der Vorstand fasst seine Beschlüsse mit einfacher Stimmenmehrheit; bei Stimmengleichheit gibt die Stimme des Vorsitzenden den Ausschlag.
			\item Den Vorsitz führt der Obmann, bei Verhinderung sein Stellvertreter. Ist auch dieser verhindert, obliegt der Vorsitz dem an Jahren ältesten anwesenden Vorstandsmitglied oder jenem Vorstandsmitglied, das die übrigen Vorstandsmitglieder mehrheitlich dazu bestimmen.
			\item Ausser durch den Tod und Ablauf der Funktionsperiode (Abs. \ref{abs:vorstand-funktionsperiode}) erlischt die Funktion eines Vorstandsmitglieds durch Enthebung (Abs. \ref{abs:vorstand-enthebung}) und Rücktritt (Abs. \ref{abs:vorstand-ruecktritt}).
			\label{abs:vorstand-ende}
			\item Die Generalversammlung kann jederzeit den gesamten Vorstand oder einzelne seiner Mitglieder entheben. Die Enthebung tritt mit Bestellung des neuen Vorstands bzw. Vorstandsmitglieds in Kraft.
			\label{abs:vorstand-enthebung}
			\item Die Vorstandsmitglieder können jederzeit schriftlich ihren Rücktritt erklären. Die Rücktrittserklärung ist an den Vorstand, im Falle des Rücktritts des gesamten Vorstands an die Generalversammlung zu richten. Der Rücktritt wird erst mit Wahl bzw. Kooptierung (Abs. \ref{abs:vorstand-wahl}) eines Nachfolgers wirksam.
			\label{abs:vorstand-ruecktritt}
		\end{enumerate}

	\section{Aufgaben des Vorstands}
		\label{sec:aufgaben-vorstand}
		Dem Vorstand obliegt die Leitung des Vereins. Er ist das ,,Leitungsorgan`` im Sinne des Vereinsgesetzes 2002. Ihm kommen alle Aufgaben zu, die nicht durch die Statuten einem anderen Vereinsorgan zugewiesen sind. In seinen Wirkungsbereich fallen insbesondere folgende:
		Angelegenheiten:
		\begin{enumerate}
			\item Einrichtung eines den Anforderungen des Vereins entsprechenden Rechnungswesens mit laufender Aufzeichnung der Einnahmen/Ausgaben und Führung eines Vermögensverzeichnisses als Mindesterfordernis,
			\item Erstellung des Jahresvoranschlags, des Rechenschaftsberichts und des Rechnungsabschlusses,
			\item Vorbereitung und Einberufung der Generalversammlung in den Fällen des § \ref{sec:generalversammlung} Abs. \ref{abs:ordentliche-generalversammlung} und Abs. \ref{abs:ausserordengliche-generalversammlung} lit. \ref{abs:ausserordentlich-rechnung} dieser Statuten,
			\item Information der Vereinsmitglieder über die Vereinstätigkeit, die Vereinsgebarung und den geprüften Rechnungsabschluss,
			\item Verwaltung des Vereinsvermögens,
			\item Aufnahme und Ausschluss von ordentlichen und unterstützenden Vereinsmitgliedern,
			\item Aufnahme und Kündigung von Angestellten des Vereins.
		\end{enumerate}

	\section{Besondere Obliegenheiten einzelner Vorstandsmitglieder}
		\label{sec:obliegenheiten-vorstand}
		\begin{enumerate}
			\item Der Obmann führt die laufenden Geschäfte des Vereins. Der Schriftführer unterstützt den Obmann bei der Führung der Vereinsgeschäfte.
			\item Der Obmann vertritt den Verein nach aussen. Schriftliche Ausfertigungen des Vereins bedürfen zu ihrer Gültigkeit der Unterschriften des Obmanns und des Kassiers oder eines Vorstandsmitgliedes. Rechtsgeschäfte zwischen Vorstandsmitgliedern und Verein bedürfen der Zustimmung eines anderen Vorstandsmitglieds.
			\label{abs:obliegenheiten-aussengeschaefte}
			\item Rechtsgeschäftliche Bevollmächtigungen, den Verein nach aussen zu vertreten bzw. für ihn zu zeichnen, können ausschliesslich von den in Abs. \ref{abs:obliegenheiten-aussengeschaefte} genannten Vorstandsmitgliedern erteilt werden.
			\item Bei Gefahr im Verzug ist der Obmann berechtigt, auch in Angelegenheiten, die in den Wirkungsbereich der Generalversammlung oder des Vorstands fallen, unter eigener Verantwortung selbständig Anordnungen zu treffen; im Innenverhältnis bedürfen diese jedoch der nachträglichen Genehmigung durch das zuständige Vereinsorgan.
			\item Der Obmann führt den Vorsitz in der Generalversammlung und im Vorstand.
			\item Der Schriftführer führt die Protokolle der Generalversammlung und des Vorstands.
			\item Der Kassier ist für die ordnungsgemässe Geldgebarung des Vereins verantwortlich.
			\item Im Fall der Verhinderung treten an die Stelle des Obmanns, des Schriftführers oder des Kassiers ihre Stellvertreter.
		\end{enumerate}

	\section{Rechnungsprüfer}
		\label{sec:rechnungspruefer}
		\begin{enumerate}
			\item Zwei Rechnungsprüfer werden von der Generalversammlung auf \funktionsperiode gewählt. Wiederwahl ist möglich. Die Rechnungsprüfer dürfen keinem Organ --- mit Ausnahme der Generalversammlung --- angehören, dessen Tätigkeit Gegenstand der Prüfung ist.
			\item Den Rechnungsprüfern obliegt die laufende Geschäftskontrolle sowie die Prüfung der Finanzgebarung des Vereins im Hinblick auf die Ordnungsmässigkeit der Rechnungslegung und die statutengemässe Verwendung der Mittel. Der Vorstand hat den den Rechnungsprüfern die erforderlichen Unterlagen vorzulegen und die erforderlichen Auskünfte zu erteilen. Die Rechnungsprüfer haben dem Vorstand über das Ergebnis der Prüfung zu berichten.
			\label{abs:rechnungspruefer-kontrolle}
			\item Rechtsgeschäfte zwischen Rechnungsprüfern und Verein bedürfen der Genehmigung durch die Generalversammlung. Im Übrigen gelten für die die Rechnungsprüfer die Bestimmungen des § \ref{sec:vorstand} Abs. \ref{abs:vorstand-ende} bis \ref{abs:vorstand-ruecktritt} sinngemäss.
		\end{enumerate}

	\section{Schiedgericht}
		\label{sec:schiedsgericht}
		\begin{enumerate}
			\item Zur Schlichtung von allen aus dem Vereinsverhältnis entstehenden Streitigkeiten ist das vereinsinterne Schiedsgericht berufen. Es ist eine ,,Schlichtungseinrichtung`` im Sinne des Vereinsgesetzes 2002 und kein Schiedsgericht nach den §§ 577 ff ZPO.
			\item Das Schiedsgericht setzt sich aus drei ordentlichen Vereinsmitgliedern zusammen. Es wird derart gebildet, dass ein Streitteil dem Vorstand ein Mitglied als Schiedsrichter schriftlich namhaft macht. Über Aufforderung durch den Vorstand binnen sieben Tagen macht der andere Streitteil innerhalb von 14 Tagen seinerseits ein Mitglied des Schiedsgerichts namhaft. Nach Verständigung durch den Vorstand innerhalb von sieben Tagen wählen die namhaft gemachten Schiedsrichter binnen weiterer 14 Tage ein drittes ordentliches Mitglied zum Vorsitzenden des Schiedsgerichts. Bei Stimmengleichheit entscheidet unter den Vorgeschlagenen das Los. Die Mitglieder des Schiedsgerichts dürfen keinem Organ --- mit Ausnahme der Generalversammlung --- angehören, dessen Tätigkeit Gegenstand der Streitigkeit ist.
			\item Das Schiedsgericht fällt seine Entscheidung nach Gewährung beiderseitigen Gehörs bei Anwesenheit aller seiner Mitglieder mit einfacher Stimmenmehrheit. Es entscheidet nach bestem Wissen und Gewissen. Seine Entscheidungen sind vereinsintern endgültig.
		\end{enumerate}

	\section{Freiwillige Auflösung des Vereins}
		\label{sec:freiwillige-aufloesung}
		\begin{enumerate}
			\item Die freiwillige Auflösung des Vereins kann nur in einer Generalversammlung und nur mit Zweidrittelmehrheit der abgegebenen gültigen Stimmen beschlossen werden.
			\item Die Generalversammlung hat --- sofern Vereinsvermögen vorhanden ist --- über die Abwicklung zu beschliessen. Insbesondere hat sie einen Abwickler zu berufen und Beschluss darüber zu fassen, wem dieser das nach Abdeckung der Passiva verbleibende Vereinsvermögen zu übertragen hat.
			\item Der letzte Vereinsvorstand hat die freiwillige Auflösung binnen vier Wochen nach Beschlussfassung der zuständigen Vereinsbehörde schriftlich anzuzeigen.
		\end{enumerate}

	\section{Selbstständige Auflösung des Vereins}
		Die selbstständige Auflösung des Vereins tritt in folgenden Fällen automatisch mit dem ersten Tag des folgenden Monats in Kraft:
		\begin{enumerate}
			\item Im Fall eines weltweiten thermonuklearen Krieges,
			\item Beim Wegfall des Vereinszweck, zum Beispiel durch:
			\begin{enumerate}
				\item das Inkrafttreten eines Gesetzes in der Republik Österreich welches den Zugriff auf das Internet verbietet,
				\item das Inkrafttreten eines Gesetzes in der Republik Österreich welches den Einsatz von Verschlüsselungen und/oder Kryptografie verbietet.
			\end{enumerate}
		\end{enumerate}
		Der letzte Vereinsvorstand hat die Auflösung binnen vier Wochen nach Eintritt bei der zuständigen Vereinsbehörde schriftlich anzuzeigen.
		\label{sec:selbstaendige-aufloesung}

	\section{Verwendung des Vereinsvermögens bei Auflösung des Vereins}
		\label{sec:verwendung-vermoegen}
		Bei Auflösung des Vereins oder bei Wegfall des bisherigen begünstigten Vereinszwecks ist das nach Abdeckung der Passiva verbleibende Vereinsvermögen, für gemeinnützige oder mildtätige Zwecke im Sinne der §§ 34 ff Bundesabgabenordnung (BAO) zu verwenden. Soweit möglich und erlaubt, soll es dabei Institutionen zufallen, die gleiche oder ähnliche Zwecke wie dieser Verein verfolgen.

	\section{Besondere Bestimmungen}
		\label{sec:besondere-bestimmungen}
		\begin{enumerate}
			\item Der Verein bedient sich für die interne Kommunikation aller zum gegenwärtigen Zeitpunkt und in der Zukunft verfügbaren Mittel der elektronischen Kommunikation.
			\item Vereinsintern gilt elektronische Post (Email) als Schriftform. Eine Einladung gilt als zugestellt, wenn sie innerhalb üblicher Fristen nicht an den Absender zurückgeschickt wurde. Darüberhinaus werden Einladungen im Internet veröffentlicht.
			\item Alle Protokolle, die Statuten, die Geschäftsordnung und sonstige Schriftstücke gelten vereinsintern als veröffentlicht, wenn sie in geeigneter Form im Internet öffentlich zugänglich gemacht wurden.
			\item Alle Anreden und Personifizierungen in diesem Dokument gelten für Männer, Frauen und sonstige gleichermassen.
		\end{enumerate}
\end{document}
